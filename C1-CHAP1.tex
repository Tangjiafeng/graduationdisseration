\chapter{绪\hskip 0.4cm 论}
\label{chap1}

\section{研究背景及意义}
随着信息时代的发展,信息安全,网络安全,系统安全在社会中的作用日益重要起来。互联网作为一个自由开放,虚拟交互的全球平台,可以使人们更便利的获取,发布信息,但与此同时,互联网与个人息息相关的的资源也受到不同程度的威胁,于是安全技术研究也蓬勃发展起来。密码学技术,是安全技术研究的基石,其实从古至今都不乏密码学的研究,自密码学从外交情报和军事领域走向公开后,社会信息流通的方式,也深刻影响着密码学的特点。传统计算机的出现,使得古典密码的破解变得容易,计算机网络的数据传输需要更安全的密码算法,于是产生了一些经典的加密算法,如:DES,AES等对称加密算法,RSA,ECC(椭圆曲线加密算法)等非对称加密算法。密码设计者与密码分析者相互竞技,共同促进密码学平衡的发展。但是,量子计算机的问世,让密码学领域的格局发生了巨大的变化。

量子计算机作为第六类计算机,使用的计算方式和平常使用的普通计算机非常不同。量子计算机使用量子位进行计算,可以将普通计算机需要执行几十年的任务在几秒钟之内完成。目前出现的一些量子算法,如Shor算法【引用】和Grover算法【引用】已经对互联网中应用广泛的 RSA 算法、ElGamal算法、ECC公钥密码算法和Diffie-Hellamn密钥协商协议进行有效的密码攻击。如此以来,经典加密算法将受到严重的威胁,虽然在短时间内量子计算机的硬件成本和理论模型实际运作的难度不会让量子计算机真正的破解已经在商用的密码算法,但是为了防范未来量子计算机的攻击,许多种防御量子计算的加密算法也在研究之中。比如:基于Hash函数的公钥密码体制、基于格问题的公钥密码体制、基于多变量问题的公钥密码体制、基于编码问题的公钥密码体制。本文主要讨论的是基于编码问题的公钥密码体制。作为抵御量子计算机攻击的算法,基于编码的加密算法的理论基础却不是来源于量子物理,它的理论来源是信息论,编码理论,代数理论等数学知识。

基于编码的加密算法,就不会受到Shor算法或者Grover算法的影响,从而保证网络通信的安全。50年代,随着C.E Shanon《通信的数学理论》的发表,信道编码定理给出了提高多类信道上传输消息的效率,采用性质良好的纠错码的指导。60年代纠错码的研究进入快速发展期,期间有广泛应用的汉明码、Reed-Muller码、BCH码、Goppa码等等。纠错码具备的检查错误或纠正错误的能力,被很好的应用到了公钥密码体制中。1978年,McEliece首次提出基于编码的公钥加密方案,采用可以快速译码的Goppa码,安全性依赖于一般线性码译码问题(NP-完全问题)。在一些已知的攻击算法中,其工作因子都是在$2 ^ {70}$以上,具备较高的安全性。其变形方案Niderreiter公钥密码体制在公钥私钥设置中有所不同,但在安全性上被证明是等价的。

在之后的研究中,基于编码的加密方案在码的选择和公钥的构造方式上做了很多尝试,目的就是为了减小公钥大小和提升算法效率,使得在实际中快速落地,才能更好的发展。综上所述,研究基于编码的加密方案具有十分重要的意义。


\section{隐私保护研究现状}
隐私保护一直是国内外学者研究的热点之一,研究内容主要有隐私保护数据的发布、用户空间位置的隐私保护。

数据发布是当前数据挖掘、数据分析到信息共享的一个重要环节。信息大爆炸以来,人们可以明显感受到大数据的来势凶猛。据相关调查显示,目前全球互联网每天的流量累计达1EB(即10亿GB 或1000PB),这意味着每天产生的信息量可刻满1.88 亿张DVD 光盘。海量数据如同一座未经开采的金矿,里面包含着无尽的信息与财富。但与此同时,也给数据的隐私带来了威胁。例如,通过对超市顾客的购买商品的记录进行分析,可以发现各种商品之间的关联(如啤酒与尿布),从而更好的进行货架的物品整理。然而在挖掘和分析的过程当中,不可避免的会使得顾客的信息暴露,从而可能造成顾客敏感信息的泄露。在文献\cite{sweeney}中,通过对性别、出生日期、住址等属性对选民登记表和隐藏了唯一标识符的医疗信息表进行连接操作,发现超过87\%的美国公民的身份可以被标识。因此,如何解决数据发布过程中存在的隐私泄露问题,已成为隐私保护研究的重点对象,也由此产生了一个新的研究领域——隐私保护数据的发布。

数据发布的隐私保护技术主要有数据加密、数据匿名、数据扰乱等隐私保护技术。数据加密技术主要是基于密码学的隐私保护技术,文献\cite{group}利用群签名算法自主生成一系列伪签名证书来达到隐私保护效果,文献\cite{homomor}利用全同态加密方法使得服务器在不知道任何明文内容的情况下可以在密文域上进行运算操作,得到的结果与在明文进行运算的结果相同。文献\cite{muticompute} 利用安全多方计算,可以保证在他人无法获得个人数据内容的情况下,计算出想要的结果。数据扰乱是一种数据失真的技术,Dwork等人提出了一种典型的数据扰乱隐私保护模型—— 差分隐私模型\cite{differential},通过对发布的数据添加噪声进行随机扰动,使得在统计意义上攻击者无论具有何种背景知识,都不能判断一条记录是否存在原始数据表中。基于数据匿名的隐私保护技术主要是通过{$k$}-匿名技术\cite{kanonomy},在一个满足{$k$}- 匿名的数据库中,对于某一个准标识符({$QID$}, Quasi-Identifiers),值相同的记录至少有{$k$}条记录,因此通过{$QID$}去推断某一个目标记录的概率最多为{$1/k$}。

用户空间位置的隐私保护旨在保护用户当前的位置,近些年来,出现了很多位置隐私保护技术,在一定程度上保护了位置的隐私。这些技术主要包括:信息访问控制(Information access control)\cite{Myles}\cite{Youssef}、混合区域(Mix zone)\cite{Beresford}、 {$k$}- 匿名技术({$k$}-anonymity)\cite{Bamba}\cite{Chow}\cite{Mokbel}、假地址技术(Dummy locations)\cite{YiuML}\cite{Shankar}、 地理数据转换(Geographic data transformation)\cite{HuH}\cite{Khoshgozaran}、 私有信息检索(Private Information Retrial,PIR)\cite{Ghinita}\cite{GhinitaPRIVE}。

基于访问控制,混合区域以及{$k$}-匿名的LBS查询需要服务提供商或者中间件维护所有用户的位置。当服务器提供商/中间件由不可信方代理,受到的保护力度会相应的降低,因此容易受到第三方的攻击。在过去,私人数据无意间就暴露在互联网上。

{$k$}-匿名最初用在身份隐私保护。将{$k$}-匿名用在位置的隐私保护有点不适当,在位置的隐私保护概念中位置之间的距离是最重要的(身份隐私保护中身份之间的间隔是重要因素)。基于{$k$}-匿名的LBS查询精度很大程度上受到移动用户的密度和分布的影响,而这一影响因素已经超过了位置隐私保护技术所能控制的范围。

基于假地址技术的LBS查询需要移动用户随机选择一组虚假位置集合,通过移动基站将虚假位置发送给LBS服务提供商并从服务商那里获得一份错误的报告。这将导致移动设备的通信和计算量过载。为了提高效率,移动用户也许减少集合中虚假位置的数量,但这将导致弱隐私性。

基于地理数据转换的LBS查询易受到访问模式攻击\cite{Williams},因为相同的查询总是返回相同的加密结果。例如,LBS服务提供商可以观察返回密文出现的频率,依靠数据库内容的相关背景知识,可以根据出现频率匹配出最有可能的明文结果,从而得到相关的查询信息。

基于PIR的LBS查询提供了很强的密码保障,通过数据加密使得服务器无法得到用户位置的信息,且能对用户的请求提供正常的服务。相比于之前的位置隐私保护技术,PIR技术对位置隐私保护更加的安全,从理论上完全杜绝了敌手的攻击。


\section{本文工作与主要贡献}
\begin{itemize}
  \item \textbf{位置隐私保护方法的对比分析} ~~罗列出当前位置隐私保护系统结构以及的常用技术,对比各个隐私保护方法,分析他们的优缺点,并适当作出改进,在基于假位置隐私保护技术中,改进现有的算法,使得隐私保护力度更强。
  \item \textbf{针对KNN查询的隐私保护算法} ~~提出了一个具有语义安全的$k$NN算法——密文数据上的$k$NN隐私保护(PP$k$NNONED)。
  \item \textbf{PP$k$NNONED方案} ~~介绍PP$k$NNONED方案,并且满足以下几个隐私需求:原始数据$D$以及其他的任何中间结果都不应该暴露给云服务提供商、查询用户Bob的查询请求$q$对云服务提供商保密、Bob除了查询$q$ 的分类标签$c_q$外,不知道其他任何信息、$c_q$仅Bob可知、数据的访问模式对Bob以及云服务提供商来说都是保密的。
  \item \textbf{隐私保护安全性分析} ~~对PP$k$NNONED方案进行安全性分析,并证明PP$k$NNONED方案具有语义安全性。
\end{itemize}
\section{组织结构}
第一章为绪论,主要介绍的是本文章的研究背景以及意义,对当下LBS的应用和LBS带来的隐私保护进行了介绍和总结,并对文章的主要工作和文章的章节进行了介绍。


第二章分别对LBS应用模式、隐私保护系统结构以及隐私保护的一些常用技术做了介绍。

第三章主要对隐私保护的基础知识进行了系统性的描述,包括$k-$NN问题、密码学基础知识、语义安全、同态加密以及Pailler公钥加密进行了系统的概述。

第四章着重介绍一种能够满足加密数据上语义安全的$k$近邻分类器——
PP$k$NNONED 方案。并介绍了协议中涉及到的一些子协议,接着对子协议的安全
性进行了相关的介绍,并给出了相关的算法步骤。最后对子协议——最小值安全
协议进行了安全性证明。

第五章详细介绍了 PP$k$NNONED 隐私保护方案,对 PP$k$NNONED 方案进行了
算法描述。对$k$近邻安全检索进行了详细描述。另外对多数类别的安全计算进行
了描述,也给出了相关的算法描述,最后对复杂性进行了分析。

第六章总结了全文,并对未来的研究工作进一步的展望。




