\chapter{相关方案介绍}
\vspace{-0.2cm}
自基于编码的加密方案提出以来,相关专家和学者不断地改进这一极有希望在后量子时代代替基于传统数论的加密算法的候选者。从McEliece方案大致确定了基于编码的密码系统的加解密流程,后续的研究在公钥的设计上做了很多工作,我们需要在不影响私钥的秘密性的同时,又能保证对码结构的混淆不会扩散差错向量的影响,保留最大限度的纠检错能力,以使加解密的效率更高而成本更低。以下先探讨几个经典的基于编码的方案。

\section{McEliece方案}
本小节要讨论的原始McEliece方案,到目前为止仍未被攻破,而且在现在,当时McEliece提出的推荐参数在安全和加解密方面同样也适用。但是有所缺憾的是,方案设计的公钥需要大的存储空间,相对于常用的RSA加密算法,公钥大小的局限十分明显。参数为$n=1024,k=524,d=101$的McEliece加密方案,公钥尺寸大概在$1024-bit$的RSA算法的260倍左右。降低公钥尺寸成为了学者研究的热点,其中Kobar和Imai提出了一种重要修正的变体\cite{Misoczki2009Compact},既保证通信率又保证CCA2安全,但是后来被代数攻击方法攻破\cite{Faug2010Algebraic, Faugere2013A, Moufek2017A}。但是后续方案在降低尺寸的同时,在安全性上都有所折扣,效率也不尽人意。原始McEliece公钥密码是采用Goppa码,先回顾一下McEliece方案的算法。

\begin{breakablealgorithm}
	\small
	\renewcommand{\algorithmicrequire}{\textbf{Input:}}
	\renewcommand{\algorithmicensure}{\textbf{Output:}}
	\caption{McEliece密钥生成算法}
	\label{alg:McElieceKeyGen}
	\begin{algorithmic}[1]
		\Require
		系统安全参数:$n,t \in N$,其中$t \ll n$。
		\Ensure
		公钥$G^{pub}$,私钥$(S,D_\mathcal{C},P)$。
		\State
		密钥生成:对于给定的参数$n$和$t$,产生下列矩阵。
		\begin{itemize}
			\item 矩阵$G$:在有限域$\mathbb{F}$上的信息位数为$k$,最小距离为$d \geq 2t + 1$的Goppa码$\mathcal{C}$的$k \times n$阶生成矩阵。
			\item 矩阵$S$:$k \times k$阶的二元随机非奇异矩阵。
			\item 矩阵$P$:$k \times n$阶的二元随机置换矩阵。
		\end{itemize}
		\State
		然后计算方案的公钥$G^{pub} = SGP$。
		\begin{itemize}
			\item 公钥:$(G^{pub},t)$。
			\item 私钥:$(S,D_\mathcal{C},P)$,有效译码算法$D_\mathcal{C}$就是所用纠错码方案的陷门。
		\end{itemize}
	\end{algorithmic}
\end{breakablealgorithm}

McEliece在公钥尺寸的表现不够好,但是加密过程十分简单,位操作的复杂度较RSA减少很多,加密算法如下:

\begin{breakablealgorithm}
	\small
	\renewcommand{\algorithmicrequire}{\textbf{Input:}}
	\renewcommand{\algorithmicensure}{\textbf{Output:}}
	\caption{McEliece加密算法}
	\label{alg:McElieceEn}
	\begin{algorithmic}[1]
		\Require
		公钥$(G^{pub},t)$,长度为$k$的明文$\mathbf{m}$。
		\Ensure
		密文$\mathbf{c}$。
		\State
		随机选择一个汉明重量为$t$的随机向量$\mathbf{e} \in \mathbb{F}^n$.
		\State
		加密,产生密文:
		
		\centering $\mathbf{c} = \mathbf{m}G^{pub} + \mathbf{e}.$
	\end{algorithmic}
\end{breakablealgorithm}

解密过程要利用到编码的纠检错机制,消除加入的差错向量的影响,进而进行解密操作。

\begin{breakablealgorithm}
	\small
	\renewcommand{\algorithmicrequire}{\textbf{Input:}}
	\renewcommand{\algorithmicensure}{\textbf{Output:}}
	\caption{McEliece解密算法}
	\label{alg:McElieceDe}
	\begin{algorithmic}[1]
		\Require
		密文$\mathbf{c}$,私钥$(S,D_\mathcal{C},P)$。
		\Ensure
		明文$\mathbf{m}$。
		\State
		解密密文$\mathbf{c}$之前,首先计算:
		\begin{center}
			$\mathbf{c}P^{-1} = \mathbf{m}SG \oplus\textbf{e}P^{-1}.$
		\end{center}

		\State
		然后对其进行译码,因为上一步的计算结果可以看成为是码的一个含有$t$个错误的码字,所以经过译码可以得到:
		\begin{center}
			$\mathbf{m}SG = D_\mathcal{C}(\mathbf{c}P^{-1}).$
		\end{center}
		
		\State
		最后,令集合$J \subseteq \{0,1,2,...,n\}$,需要通过矩阵变换使$G_j^{pub}$可逆,则进行如下计算可以得到明文。
		\begin{center}
			$\mathbf{m} = (\mathbf{m}SG)_j(G_j)^{-1}S^{-1}.$
		\end{center}		
	\end{algorithmic}
\end{breakablealgorithm}

McEliece方案在安全性上的表现是有优势的\cite{Kobara2003On, OnTheMc, 2005纠错码的代数理论},是目前为止仍未被攻破的公钥加密算法。在已知的密码分析方法中,比如区分者攻击、信息集攻击等,针对McEliece的攻击的工作因子大多在$2^{70}$以上。在讨论改进的密码分析方法之前,我们首先来看在一些原始的密码分析方法在原始McEliece方案的工作因子大小。

\begin{table}[H]
\centering
\setlength{\tabcolsep}{8mm}{
	\begin{tabular}{l|c}
		\hline
		攻击方法 & 工作因子 \\
		\hline
		猜测私钥矩阵 & $\prod_{i=1}^{k - 1}(2^k - 2^i) + n! \approx 8.46 \times 10^{82654} + 1.85 \times 10^{2638}$ \\
		猜测码字 & $2^k \approx 5.49 \times 10^{157}$ \\
		伴随式解码攻击 & $\sum_{i=0}^{t}\binom{n}{i} \approx 3.36 \times 10^{85}$ \\
		信息集攻击 & $k^3(1 - \frac{t}{n})^{-k} \approx 3.55 \times 10^{19}$ \\
		\hline
    \end{tabular}
    }
\caption{常见攻击工作因子}\label{wf_table}
\end{table}
综上可以发现,McEliece方案在推荐参数下,直接攻击方法是无法影响其安全性的。后续的密码分析研究,在以上攻击的基础上进行了改进,使得工作因子尽可能下降,这也间接的导致了McEliece方案加强安全性的设计。本章最后会介绍一种新的加密方案变体。

\section{Niederreiter方案}
1986年,Niederreiter对McEliece公钥密码方案做出改进,提出了一种Niederreiter公钥密码方案。该密码方案基于的困难问题也是随机线性码的译码困难问题,只是利用的角度有所不同。Niederreiter密码方案隐藏了Goppa码的校验矩阵,于是在公钥尺寸上有所减少,但是还是没有达到实用的要求。

\begin{breakablealgorithm}
	\small
	\renewcommand{\algorithmicrequire}{\textbf{Input:}}
	\renewcommand{\algorithmicensure}{\textbf{Output:}}
	\caption{Niederreiter方案}
	\label{alg:Niederreiter}
	\begin{algorithmic}
		\State
		系统安全参数:$n,t \in N$,其中$t \ll n$。
		\State
		\textbf{密钥生成阶段:}
		
		\begin{itemize}
			\item 矩阵 $H$:在有限域$\mathbb{F}$上的信息位数为$k$,最小距离为$d \geq 2t + 1$的Goppa码$\mathcal{C}$的$(n-k) \times n$阶校验矩阵。
			\item 矩阵 $A$:随机选取的$(n-k) \times (n-k)$可逆矩阵.
			\item 矩阵 $P$:随机选取的$n \times n$阶的置换矩阵。			
		\end{itemize}
	
		然后计算方案的公钥:$H^{pub} = AHP$,于是公钥为$(H^{pub}, t)$,私钥为$(A,D_\mathcal{C},P)$,$D_\mathcal{C}$是码的伴随式译码算法。
		
		\State
		\textbf{加密阶段:}
		
		Niederreiter首先将明文字符串映射成一个错误向量$\mathbf{e}$,重量为$t$。加密过程即计算一个伴随式,如下:
		
		\begin{center}
			$s = H^{pub}\mathbf{e}^\mathtt{T}.$
		\end{center}
		
		加密者向解密者发送密文$s$。
		\State
		\textbf{解密阶段:}
		
		为了解密密文,首先计算:
		\begin{center}
			$A^{-1}s = HP\mathbf{e}^\mathtt{T}.$
		\end{center}
	    利用码的伴随式译码算法$D_\mathcal{C}$恢复出$P\mathbf{e}^\mathtt{T}$,于是得到明文$\mathbf{e}^\mathtt{T} = P^{-1}P\mathbf{e}^\mathtt{T}$。		
	\end{algorithmic}
\end{breakablealgorithm}

Niederreiter密码方案公钥尺寸更小,而安全性上被认为与McEliece密钥方案一致,李元兴等人从针对Niederreiter方案的攻击分析了其安全性\cite{李元兴1993关于}。但是在将明文映射称错误向量的操作上,会影响加解密的效率。Niederreiter密码方案为了抵抗经典的信息集译码攻击,往往将参数设置的很大,故还不太适合实际应用。

后续也有大量研究Niederreiter的变形方案,其中Maurich等人提出的混合加密\cite{Maurich2016IND}可以达到IND-CCA安全,而且选择QC-LDPC码可以在存储公钥的时候大大减少公钥的尺寸。
\section{BBCRS方案}
2014年,Marco Baldi、Marco Bianchi、France Chiaraluce、Joachim Rosenthal、Davide Schipani等人设计了一种提升和加强McEliece安全性的方案,简称“BBCRS方案”。之前及后续的McEliece方案变体的核心工作,就是更好的隐藏密钥结构,但是其纠错检错能力又不会大打折扣,如此就可以更好的避免被攻击成功的概率\cite{Baldi2017Post}。BBCRS方案,从原始McEliece方案的置换矩阵入手,考虑到置换矩阵暴露的安全性问题,采用一种由两个矩阵相加的形式代替置换矩阵,让多种经典的攻击手法找不到合适的入口,简言之,就是使得私钥结构与公开信息之间不具备置换转化的关系。BBCRS方案的设计主要贡献在于密钥的生成,私钥结构隐藏更好就可以换取结构性更强的码,以带来公钥的紧凑,尺寸减小的优势。BBCRS增加了一些系统参数,可以根据实际情况调整,以适应安全性,加解密效率,存储效率的平衡。

BBCRS的密钥生成与原始的McEliece有很大的区别,且组成过程些许复杂。假设两个包含$w$个矩阵元素的集合$\mathcal{A}$和$\mathcal{B}$,内部元素都是随机选择并且保密的$z \times n, z \leq n$矩阵,而矩阵元素都在有限域$\mathbb{F}_q$上。
\begin{center}
	$\mathcal{A} = \{\mathbf{a}_1,\mathbf{a}_2,...,\mathbf{a}_w\},~$$~\mathcal{B} = \{\mathbf{b}_1,\mathbf{b}_2,...,\mathbf{b}_w\}.$
\end{center}
\begin{flushleft}
	令矩阵$R = \mathbf{a}_1^\mathtt{T}\mathbf{b}_1 + \mathbf{a}_2^\mathtt{T}\mathbf{b}_2 + \cdots + \mathbf{a}_w^\mathtt{T}\mathbf{b}_w$,则矩阵$R$就是一个密集的$n \times n$矩阵。另一方面,定义一个在有限域$\mathbb{F}_q$上的$n \times n$稀疏矩阵$T$,矩阵$T$的平均行重和列重都是一个常数$m,~ m \ll n$。当$m$是一个整数时,又叫重量为$m$的广义置换矩阵,$m$也可以不是整数,在BBCRS方案中,$m$的值是在$[1,2]$范围中的。
\end{flushleft}

最后经过矩阵$R$和$T$的加法计算,就是最终应用到公钥设计的矩阵$Q$。即$Q=R+T$,在原始McEliece方案中替换掉置换矩阵$P$。在该种构造方式下,BBCRS方案提出了两套参数设定,在$w = 2$前提下,\ding{172}~$\mathbf{a}_2 = \mathbf{0}$;\ding{173}$\mathbf{b}_2=\mathbf{1} + \mathbf{b}_1$。其中$\mathbf{0}$和$\mathbf{1}$分别指的是全零矩阵和全$1$矩阵。在此设定下,保证矩阵$R$的秩为$z$。

在进行加解密之前,关于加密过程中随机产生的差错向量还要满足一些性质,才能被正确的解码。对于差错向量$\mathbf{e}$,预定$wt(\mathbf{e}) \leq \lfloor \frac{t}{m} \rfloor$。其中$m$是矩阵$T$的平均行重或列重,$t$是纠错码的最大纠错个数。而且差错向量$\mathbf{e}$还有如下限制:
\begin{equation}
	(\mathbf{a}_1 + \mathbf{a}_2 + ··· + \mathbf{a}_w) \cdot \mathbf{e}^\mathtt{T} = \mathbf{0}.
\end{equation}

\begin{flushleft}
	BBCRS的详细算法过程如下:
\end{flushleft}
\begin{breakablealgorithm}
	\small
	\renewcommand{\algorithmicrequire}{\textbf{Input:}}
	\renewcommand{\algorithmicensure}{\textbf{Output:}}
	\caption{BBCRS公钥密码体制McEliece版本}
	\label{alg:BBCRS}
	\begin{algorithmic}	
		\State
		系统安全参数:$n \in N, t \in N, m, z \in N, w \in N$,其中$t,m \ll n$。
		\State
		\textbf{密钥生成阶段:}
		
		\begin{itemize}
			\item 矩阵 $G$:在有限域$\mathbb{F}$上的信息位数为$k$,最小距离为$d \geq 2t + 1$的Goppa码$\mathcal{C}$的$k \times n$阶生成矩阵。
			\item 矩阵 $S$:随机选取的$k \times k$可逆矩阵。
			\item 矩阵 $Q$:如上文所述,由$R + T$计算得出。
		\end{itemize}
		
		然后计算方案的公钥:$G^{pub} = S^{-1}GQ^{-1}$,于是公钥为$(G^{pub}, t, m)$,私钥为$(S,D_\mathcal{C},R,T)$,$D_\mathcal{C}$是码的译码算法。
		
		\State
		\textbf{加密阶段:}
		
		首先随机选取一个满足$wt(\mathbf{e}) \leq \lfloor \frac{t}{m} \rfloor$和式 3.1的差错向量$\mathbf{e}$,对明文$\mathbf{m} \in \mathbb{F}^k$计算:
		
		\begin{center}
			$\mathbf{c} = \mathbf{m}G^{pub} + \mathbf{e}.$
		\end{center}
		
		加密者向解密者发送密文$\mathbf{c}$。
		\State
		\textbf{解密阶段:}
		
		为了解密密文,首先计算:
		\begin{center}
			$\mathbf{c}Q = \mathbf{m}S^{-1}G \oplus\mathbf{e}Q = \mathbf{m}S^{-1}G \oplus\mathbf{e}(R + T).$
		\end{center}
		
		接下来的问题就是对$\mathbf{c}Q$进行译码,重点就在于$\mathbf{e}(R + T)$的差错位数是否在码的有效纠错能力之内。首先讨论$\mathbf{e}R$,在上文提到两种设定下,根据式 3.1,$\mathbf{e}R$结果分别是:
		\begin{center}
			\ding{172}$~\mathbf{a}_2 = \mathbf{0},~\mathbf{e}R = 0,$
			
			\ding{173}$~\mathbf{b}_2=\mathbf{1} + \mathbf{b}_1,~\mathbf{e}R = \mathbf{e} \cdot \mathbf{a}^\mathtt{T} \cdot \mathbf{1}.$
		\end{center}
		当$\mathbf{e}R = 0$时,$\mathbf{e}Q = \mathbf{e}T$,由矩阵$T$的平均行重或列重以及差错向量$\mathbf{e}$的重量范围可知,$\mathbf{e}T$可以被正确译码。而针对$\mathbf{e}R \neq 0$的情况,BBCRS决定在可能值的范围里穷举搜索,直到找到一个值$\lambda = \mathbf{e}R$,令$\mathbf{e}(R + T) - \lambda$,从而可以正确译码,并计算出$\mathbf{m}S^{-1} = D_\mathcal{C}(\mathbf{c}Q)$,最后有私钥矩阵$S$右乘,还原明文$\mathbf{m}$。
	\end{algorithmic}
\end{breakablealgorithm}

观察BBCRS公钥密码体制的公钥构造可知,矩阵$Q$对私钥进行隐藏和混淆,使得公钥与私钥之间不存在原始McEliece方案中的置换关系,从而防止了利用公私钥之间置换关系进行的区分攻击。BBCRS公钥密码体制在安全性和加解密效率的表现与矩阵$Q$的选择,矩阵$T$的选择息息相关。具体来说就是,如何保证矩阵$Q$对差错向量$\mathbf{e}$的计算作用不会扩散错误的个数,从而导致解码失败。这就涉及矩阵$R$的构造矩阵集合$\mathcal{A}$和$\mathcal{B}$中元素的关系设定,可以使得$\mathbf{e}R = 0$或者在解码过程可以消除$\mathbf{e}R$的影响。这其中还牵涉到对差错向量的一些限制,一方面是其最大重量减小,一方面要满足式 3.1。式 3.1在BBCRS的文章中做了分析,这是存在子码攻击的威胁的,通过计算可知公钥的一部分子码在式 3.1的作用下,矩阵$R$的混淆作用完全消失,也就是说式 3.1可以让公钥的子码与私钥的子码回归到置换相等的关系,这就是方案的私钥秘密优势荡然无存。BBCRS也给出了解决的办法,但是期间涉及一些穷举和猜测,虽然可以避免公私钥子码之间置换的关系,但是加解密的效率和信息利用率不会很高。

矩阵$T$是另一个影响解码成功率的因素,矩阵$T$可以是一个广义的置换矩阵,与此同时为了加密过程中错误向量的错误位数不超过允许的范围内,对随机选取的差错向量的错误个数做了降低。但是当$m$不是整数时,矩阵$T$的平均行重和列重就无法保证$\mathbf{e}T$的错误个数不会超过解码能力。当差错向量的非零位在计算$\mathbf{e}T$过程中集中作用在行重大于$1$的行,计算结果非零位的总和就可能超过码的纠错能力$t$。于是,矩阵$T$的选择下,存在着解码失败的概率。

BBCRS也提出了Niederreiter版本的公钥密码体制。具体流程如下:

\begin{breakablealgorithm}
	\small
	\renewcommand{\algorithmicrequire}{\textbf{Input:}}
	\renewcommand{\algorithmicensure}{\textbf{Output:}}
	\caption{BBCRS公钥密码体制Niederreiter版本}
	\label{alg:BBCRS}
	\begin{algorithmic}	
		\State
		系统安全参数:$n \in N, t \in N, m, z \in N, w \in N$,其中$t,m \ll n$。
		\State
		\textbf{密钥生成阶段:}
		
		\begin{itemize}
			\item 矩阵 $H$:在有限域$\mathbb{F}$上的信息位数为$k$,最小距离为$d \geq 2t + 1$的Goppa码$\mathcal{C}$的$(n-k) \times n$阶校验矩阵。
			\item 矩阵 $S$:随机选取的$k \times k$可逆矩阵。
			\item 矩阵 $Q$:如上文所述,由$R + T$计算得出。
		\end{itemize}
		
		然后计算方案的公钥:$H^{pub} = S^{-1}GQ^\mathtt{T}$,于是公钥为$(H^{pub}, t, m)$,私钥为$(S,D_\mathcal{C},R,T)$,$D_\mathcal{C}$是码的译码算法。
		
		\State
		\textbf{加密阶段:}
		
		首先将明文字符串映射成一个错误向量$\mathbf{e}$,满足$wt(\mathbf{e}) \leq \lfloor \frac{t}{m} \rfloor$和式 3.1,加密过程即计算一个伴随式,如下:
		
		\begin{center}
			$\mathbf{c} = H^{pub}\mathbf{e}^\mathtt{T}.$
		\end{center}
		
		加密者向解密者发送密文$\mathbf{c}$。
		\State
		\textbf{解密阶段:}
		
		为了解密密文,首先利用私钥矩阵$S$计算:
		\begin{center}
			$\mathbf{c}' = S \cdot \mathbf{c} = H \cdot Q^\mathtt{T} \cdot \mathbf{e}^\mathtt{T} = H \cdot (\mathbf{e} \cdot Q)^\mathtt{T}.$
		\end{center}
		
		前文提到的矩阵$Q$的特殊参数设定,可以让$\mathbf{e} \cdot Q$简化到$\mathbf{e} \cdot T$。此时解密者获得$H \cdot T^\mathtt{T} \cdot \mathbf{e}^\mathtt{T}$,因为 $T^\mathtt{T} \cdot \mathbf{e}^\mathtt{T}$的重量小于等于码$\mathcal{C}$的纠错能力$t$,接着进行伴随式解码,并利用私钥矩阵$T$还原由明文映射成的错误向量$\mathbf{e}$,再次反映射,得到明文$m$。
	\end{algorithmic}
\end{breakablealgorithm}
该版本的BBCRS密码体制,安全性与McEliece版本等同,在公钥尺寸上可以是公钥矩阵转换成系统形式,从而在真正存储公钥的时候,只存储其中的一部分。具体可以参考BBCRS的描述。

\section{本章小结}
本章介绍的方案是预备知识和基础编码知识在公钥加密方案中的应用。纠错码理论帮助我们加密明文,并能根据纠错码性质从密文恢复明文,码的纠错性能,生成矩阵和校验矩阵的结构性质是我们设计加密方案重要考虑因素。

在本章我们对经典的基于编码的公钥加密方案做了基本的介绍,在加解密的讲解中我们简要分析了公钥设计,密码系统安全性的变化。McEliece加密方案与Niederreiter对偶变形在之后的发展中有很多本文未涉及的具备鲜明特点的变体设计,但总的来说无外乎两大努力的方向,降低公钥尺寸与提高解码效率。在变体方案的改进思想中,我们逐渐沉淀了一些经验,比如可逆矩阵$S$的地位就是稳定不变的,私密的生成矩阵会被完全混淆在公钥矩阵的内部;另一方面,我们也看到了一些突破,置换矩阵$P$的地位已经不再稳固,需要被混淆作用更好的矩阵结构代替。

BBCRS中的私钥矩阵$Q$的设计提供了很好的改进变体的思路,后续对差错向量的控制和解密流程的调整也告诉我们,混淆私钥矩阵的右乘部分的改变,将会对整个公钥密码体制产生巨大的影响,如果处理不好,要么会带来密码分析可利用的漏洞,要么在解码效率上有所折扣。在本章的讨论中,密码攻击手段的进步也是我们方案在改进中要考虑的因素,我们发现,不管是原始的方案直接采用置换矩阵,还是BBCRS公钥密码体制的矩阵$Q$包含的类置换矩阵的矩阵$T$,总是会给一些攻击可乘之机,接下来的方案改进,十分有必要考虑这些问题并予以解决。