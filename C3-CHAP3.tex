\chapter{经典方案介绍}
\vspace{-0.2cm}
自基于编码的加密方案提出以来,相关专家和学者不断地改进这一极有希望在后量子时代代替基于传统数论的加密算法的候选者。从McEliece方案大致确定了基于编码的密码系统的加解密流程,后续的研究在公钥的设计上做了很多工作,我们需要在不影响私钥的秘密性的同时,又能保证对码结构的混淆不会扩散差错向量的影响,保留最大限度的纠检错能力,以使加解密的效率更高而成本更低。以下先探讨几个经典的基于编码的方案。
\section{McEliece方案}
本小节要讨论的原始McEliece方案,到目前为止仍未被攻破,而且在现在,当时McEliece提出的推荐参数在安全和加解密方面同样也适用。但是有所缺憾的是,方案设计的公钥需要大的存储空间,相对于常用的RSA加密算法,公钥大小的局限十分明显。参数为$n=1024,k=524,d=101$的McEliece加密方案,公钥尺寸大概在$1024-bit$的RSA算法的260倍左右。降低公钥尺寸成为了学者研究的热点,但是后续方案在降低尺寸的同时,在安全性上都有所折扣,效率也不尽人意。原始McEliece公钥密码是采用Goppa,先回顾一下McEliece方案的算法。

\begin{breakablealgorithm}
	\small
	\renewcommand{\algorithmicrequire}{\textbf{Input:}}
	\renewcommand{\algorithmicensure}{\textbf{Output:}}
	\caption{McEliece密钥生成算法}
	\label{alg:McElieceKeyGen}
	\begin{algorithmic}[1]		
		\Require
		系统安全参数:$n,t \in N$,其中$t << n$。
		\Ensure
		公钥$G^{pub}$,私钥$(S,D_\mathcal{C},P)$。
		\State
		密钥生成:对于给定的参数$n$和$t$,产生下列矩阵。
		\begin{itemize}
			\item 矩阵$G$:在有限域$\mathbb{F}$上的信息位数为$k$,最小距离为$d \geq 2t + 1$的Goppa码$\mathcal{C}$的$k \times n$阶生成矩阵。
			\item 矩阵$S$:$k \times k$阶的二元随机非奇异矩阵。
			\item 矩阵$P$:$k \times n$阶的二元随机置换矩阵。
		\end{itemize}	    
		\State
		然后计算方案的公钥$G^{pub} = SGP$。
		\begin{itemize}
			\item 公钥:$(G^{pub},t)$。
			\item 私钥:$(S,D_\mathcal{C},P)$,有效译码算法$D_\mathcal{C}$就是所用纠错码方案的陷门。
		\end{itemize}
	\end{algorithmic}
\end{breakablealgorithm}

McEliece在公钥尺寸的表现不够好,但是加密过程十分简单,位操作的复杂度较RSA减少很多,加密算法如下:

\begin{breakablealgorithm}
	\small
	\renewcommand{\algorithmicrequire}{\textbf{Input:}}
	\renewcommand{\algorithmicensure}{\textbf{Output:}}
	\caption{McEliece加密算法}
	\label{alg:McElieceEn}
	\begin{algorithmic}[1]		
		\Require
		公钥$(G^{pub},t)$,长度为$k$的明文$\mathbf{m}$。
		\Ensure
		密文$\mathbf{c}$。
		\State
		随机选择一个汉明重量为$t$的随机向量$\mathbf{e} \in \mathbb{F}^n$.
		\State
		加密,产生密文:
		
		\centering $\mathbf{c} = \mathbf{m}G^{pub} + \mathbf{e}.$
	\end{algorithmic}
\end{breakablealgorithm}

解密过程要利用到编码的纠检错机制,消除加入的差错向量的影响,进而进行解密操作。

\begin{breakablealgorithm}
	\small
	\renewcommand{\algorithmicrequire}{\textbf{Input:}}
	\renewcommand{\algorithmicensure}{\textbf{Output:}}
	\caption{McEliece解密算法}
	\label{alg:McElieceDe}
	\begin{algorithmic}[1]		
		\Require
		密文$\mathbf{c}$,私钥$(S,D_\mathcal{C},P)$。
		\Ensure
		明文$\mathbf{m}$。
		\State
		解密密文$\mathbf{c}$之前,首先计算:
		\begin{center}
			$\mathbf{c}P^{-1} = \mathbf{m}SG \oplus\textbf{e}P^{-1}.$
		\end{center}		

		\State
		然后对其进行译码,因为上一步的计算结果可以看成为是码的一个含有$t$个错误的码字,所以经过译码可以得到:
		\begin{center}
			$\mathbf{m}SG = D_\mathcal{C}(\mathbf{c}P^{-1}).$
		\end{center}
		
		\State
		最后,令集合$J \subseteq \{0,1,2,...,n\}$,需要通过矩阵变换使$G_j^{pub}$可逆,则进行如下计算可以得到明文。
		\begin{center}
			$\mathbf{m} = (\mathbf{m}SG)_j(G_j)^{-1}S^{-1}.$
		\end{center}		
	\end{algorithmic}
\end{breakablealgorithm}

McEliece方案在安全性上的表现是有优势的,可以达到INA-CCA安全。在已知的密码分析方法中,比如区分攻击、信息集攻击等,针对McEliece的攻击的工作因子大多在$2^{70}$以上。

\section{Niederreiter方案}
1986年,Niderreiter对McEliece公钥密码方案做出改进,提出了一种Niederreiter公钥密码体制。该密码体制基于的困难问题也是随机线性码的译码困难问题,只是利用的角度有所不同。Niderreiter密码体制隐藏了Goppa码的校验矩阵,于是在公钥尺寸上有所减少,但是还是没有达到实用的要求。

\begin{breakablealgorithm}
	\small
	\renewcommand{\algorithmicrequire}{\textbf{Input:}}
	\renewcommand{\algorithmicensure}{\textbf{Output:}}
	\caption{Niederreiter公钥密码体制}
	\label{alg:NiederreiterKeyGen}
	\begin{algorithmic}	
		\State
		系统安全参数:$n,t \in N$,其中$t << n$。
		\State
		\textbf{密钥生成阶段:}
		
		\begin{itemize}
			\item 矩阵 $H$:在有限域$\mathbb{F}$上的信息位数为$k$,最小距离为$d \geq 2t + 1$的Goppa码$\mathcal{C}$的$(n-k) \times n$阶校验矩阵。
			\item 矩阵 $A$:随机选取的$(n-k) \times (n-k)$可逆矩阵.
			\item 矩阵 $P$:随机选取的$n \times n$阶的置换矩阵。			
		\end{itemize}
	
		然后计算方案的公钥:$H^{pub} = AHP$,于是公钥为$(H^{pub}, t)$,私钥为$(A,D_\mathcal{C},P)$,$D_\mathcal{C}$是码的伴随式译码算法。
		
		\State
		\textbf{加密阶段:}
		
		Niederreiter首先将明文字符串映射成一个错误向量$\mathbf{e}$,重量为$t$。加密过程即计算一个伴随式,如下:
		
		\begin{center}
			$s = H^{pub}\mathbf{e}^\mathtt{T}.$
		\end{center}
		
		加密者向解密者发送密文$s$。
		\State
		\textbf{解密阶段:}
		
		为了解密密文,首先计算:
		\begin{center}
			$A^{-1}s = HP\mathbf{e}^\mathtt{T}.$
		\end{center}
	    利用码的伴随式译码算法$D_\mathcal{C}$恢复出$P\mathbf{e}^\mathtt{T}$,于是得到明文$\mathbf{e}^\mathtt{T} = P^{-1}P\mathbf{e}^\mathtt{T}$。		
	\end{algorithmic}
\end{breakablealgorithm}

Niederreiter密码体制公钥尺寸更小,而安全性上被认为与McEliece密钥方案一致。但是在将明文映射称错误向量的操作上,会影响加解密的效率。Niederreiter密码体制为了抵抗经典的信息集译码攻击,往往将参数设置的很大,故还不太适合实际应用。

后续也有大量研究Niederreiter的变形方案,其中Maurich等人提出的混合加密可以达到IND-CCA安全,而且选择QC-LDPC码可以在存储公钥的时候大大减少公钥的尺寸。
\section{BBCRS方案}


