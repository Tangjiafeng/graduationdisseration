{\kaishu
\chapter*{致\qquad 谢}
从2014年9月进入密码与网络安全系至今,这两年半的时光转眼即逝,在华东师范大学的研究生生活无疑是我人生中宝贵的财富之一,在这两年半的时间中我学会了如何去发现问题,独立思考问题,如何将所学知识活学活用,让我对自己有了更加深刻的了解,如何更好的规划未来的人生道路。在毕业论文即将完成之际,我要对每一位支持我,学习上鼓励我、教导我,生活上关心我的人表示衷心的感谢。

首先,我要感谢密码与网络安全系的所有老师们,老师们具有一流的科研水平以及极高科研热情,带领我们以当下最先进,最前沿的眼光来审视当下各种大数据安全,云计算安全等问题,如今网络安全被列为教育部一级学科,从斯诺登事件到如今的电信你诈骗案,处处都体现了当今密码与网络安全的重要性。曹珍富老师是密码与安全方面的专家,在学术界很有名气,我们密码与网络安全系2014年刚成立,很高兴在曹老师的带领下,队伍不断壮大,我们系也受到越来越多国内外学者的关注,引进了许多安全方向的专家学者,研究成果如雨后春笋,我感动由衷的欣慰,感谢老师们为了系部的发展做出的各种努力,感谢董老师在研究生期间对我的教导,董老师强大的专业背景让我认识到了密码学的魅力之处,总能从董老师的耐心讲解中找到正确的求解思路,感谢董老师在研究室期间对我的信任与包容,不管是在个人发展以及学术研究上,董老师总能给我最大的帮助,理解与关心。感谢周俊老师,周俊老师是我们实验室的小老师同时也是我们的大师兄,有着老师的严谨以及师兄的担当,周老师在学术研究方面有很深的造诣,每次例会都能在不懂之处给我们进行耐心的讲解与指导。感谢沈佳辰老师,沈老师为实验室付出了很多,对实验室同学们的生活、学习都很关心。感谢何道敬老师、张磊老师、曾鹏老师、王高丽老师等等,非常感谢每一位老师在过去的两年多时间对我的培养。

其次感谢所有TDT实验室陪伴我走过研究生生涯的师兄弟们,在这里对我来说就像是一个大家庭,两年多的时间,在这里见证了大家的喜怒哀乐,感谢宁建廷博士、王海江博士、李冬梅博士,郭莹博士,巩俊卿博士,曹楠源博士在我刚来实验室时对我的学术指导,感谢王丹,陈冬冬和我一起三足鼎立坚守实验室,感谢邓尔冬,张华君、来思远,赵晓鹏,郑锦文,毋萌,宋春芝,张晓东,王乾,郭婉芬,丁诗瑶等在TDT实验室学习的师兄姐弟妹们,感谢密码与网络安全系所有的同学们。还要感谢这两年多来包容我,陪伴我度过最美好的研究生生活的“逗逼”舍友们。

此外,我想感谢所有软件学院的老师和朋友,感谢陪我走过两年多研究生生活的的软件专硕班,在这个班集体中我遇到了积极向上,执着努力热爱生活的一帮程序员么。

最后,感谢我的父母,他们是我背后最坚强的后盾。虽然离家不算远,可每年只能在家陪伴他们短短的天数,每次回家,爸妈都会为我准备丰富的饭菜,从来不会因为我没有在他们身边而有多抱怨。他们教会了我成为一个正直,勤奋,有担当的人,总是默默的支持我的任何想多的事情,无私的付出。希望在将来不会辜负他们对我的期望,对得起父母的养育之恩。

\vspace{0.8cm} \hspace{9.8cm}  孙\,浩

\hspace{9cm}  二零一七一月 }
