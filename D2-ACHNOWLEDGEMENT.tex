{\kaishu
\chapter*{致\qquad 谢}
很荣幸在华东师范大学度过了最有价值的学习和生活时光,回想起考研的初衷,我在读研期间总是感到莫大的幸福,这里有丰富的学术资源,良好的学习环境,先进的图书馆,每次从图书馆出来都带着收获知识的成就感。天气晴朗的傍晚,我总是到共青操场上做有氧运动,锻炼身体的同时,也放松了学业的紧张,入学以来,华东师范大学一直以开放,严谨的态度影响着我们的成长,在毕业之际,对学校,对老师,对同学,对自己的努力和收获,表示真诚的感谢。

在学校的学习生涯总是充足而美好,我们有大量的时间去研究我们感兴趣的知识,新兴的技术,还有重点加强自己的专业课背景知识,这是一个弥足珍贵的经历。我们有老师授课解惑,有同学可以第一时间碰撞大脑中的想法,提升学习的效率。其中我的导师,曾鹏副教授,就是给我学术带来深刻的学术启发和治学的严谨态度,没没想起讨论版上曾老师严丝合缝的逻辑推理,深入浅出的理论讲解,我都惊讶和敬佩,这就是科研的魅力。曾老师对我的帮助是巨大的,论文的改进,为人处世的品质,都是对我人生产生重大影响的重要部分。

还有我们的各位授课老师,感谢我系的系主任曹珍富老师,学术讲座在曹老师的主持下总是让人印象深刻,收获满满。感谢张磊老师带我们学习基础的密码学知识,感谢王高丽老师,李成举老师带我们深入科学的研究方法,带给我们吸收科研知识的能力,开阔了我们的眼界。还要感谢辅导员张炜帆老师,张晓雅老师,感谢教务处吴玲颖老师,张文清老师,她们在学校行政上,学生工作上帮助我们顺利的完成各项任务,并最终完成毕业要求。学习生活当中,相处最多的还是我们那些志趣相投的伙伴们,他们活跃的学术思想,快速的学习能力,坦诚待人的个性,让我觉得在学校每天都很快乐。在学习上多亏陈思远,沈祖明,钱于引师兄师姐的引导,感谢黄悦,黄志刚,桂斌,陆婷婷在课程上,研究方向上的讨论启发,感谢张志婷,赵景给予我新的思路。

作为读书走出来的孩子,我感恩我的父母,给予我充分的读书时间,给予我宽松的成长环境,我在做决定时,总会想起父母信任我,鼓励我的样子,这让我在人生的路途中充满力量。感谢我的另一半,徐孟金,在我困惑的时候开导我,挫折的时候陪伴我,给我开心的日常,给我稳定的满足,让我更加专注的发展自己,提高自己。

\vspace{0.8cm} \hspace{9.8cm}  周\,玉\,壮

\hspace{9cm}  二零一九十月 }
