{\kaishu
\chapter*{致\qquad 谢}
时间犹如白驹过膝,转瞬即逝,两年半的硕士研究生生涯将要告一段落。在华东师范大学学习生活的两年半的时间中,有太多太多温馨的事情值得我去留念,太多太多美好的事情让我难以忘怀。依稀记得自己第一次踏入校门激动的样子,也记得和同学漫步在丽娃河畔的情景,师大的一草一木,一楼一阁都在印在我的脑海里。在此毕业论文完成之时,向陪伴和关心我的导师、同学和朋友们表示最诚挚和衷心的感谢。

师大的求学之路来之不易,特别感谢我的导师,曾鹏副教授,感谢带领我进入这个殿堂,他严肃的科研态度,严谨的治学精神,渊博的专业知识,平易近人的人格魅力,深深的感染和鼓舞着我。回首研究生期间的生活,曾老师给了我太多太多的帮助。进入实验室以来,先后参与了密码学和抗量子密码相关课题和项目的研究,都离不开曾老师一直的指导和陪伴。在曾老师的悉心指导和帮助下,我才能克服科研阶段遇到的种种问题,取得了一些成果。在此谨向曾老师致以最诚挚的感谢和最崇高的敬意。在人生的道路上,我会始终牢记曾老师的谆谆教诲,严格要求自己。

感谢密码与网络安全系的曹珍富老师、张磊老师、王高丽老师、李成举老师、沈佳辰老师、周俊老师等,是你们的辛勤付出,才让我学到了先进的科研知识,拓宽了我的眼界高度。感谢你们,让我对科研学习产生了深厚的兴趣,帮助我解决科研阶段一个又一个问题。还要感谢辅导员张炜帆老师在生活上、学习上对我的关心和照顾,还有教务秘书吴玲颖老师、孙玉华老师在教务工作中、学生事务上给予我们的帮助。

感谢陈思远,沈祖明,钱于引三位前辈在科研学习上给我树立的榜样,也一直在帮助、鼓励着我。感谢黄悦、黄志刚、桂斌、鲁婷婷在论文撰写阶段对我的照顾。感谢赵景、张志亭给予我论文实验阶段的辛勤付出。感谢实验室的其他小伙伴们,有了你们的存在,我的研究生生活才会如此多姿多彩。感谢父母和家人的无私关怀,无论是在学习上还是生活上,是你们在我背后一直默默的支持我、鼓励我,你们永远是我最坚持的后盾,今生无以为报。


\vspace{0.8cm} \hspace{9.8cm}  周\,玉\,壮

\hspace{9cm}  二零一九十月 }
