\chapter{总结与展望}
\section{总结}
基于编码的加密方案,一直是后量子时代竞争基于传统数论的加密方案的主要方案,立足于信息论,编码理论,有限域理论,基于随机线性码的译码问题,Goppa码的不可区分性问题两大困难假设,历经四十余年的发展,理论上是迄今为止安全性最高公钥加密方案。然而在实际应用领域中,还是传统的对称密码与基于传统数论的公钥密码结合使用。对于私钥共享,私钥传输等小体量的信息采用对称加密的高效方式,但是对称加密的密钥管理会比较浪费存储资源,在网络通信或其它不可信信道中,传输大体量数据主要还是采用公钥加密方案。而我们的基于编码的公钥加密方案,一直受制于其公钥尺寸问题,实际的参数下,基于编码的的加密方案的公钥往往是RSA公钥密码的几百倍,这显然不适用于实际的应用中。


自McEliece方案提出以来,其安全性等级一直是比较高的,但在攻击手段不断的更新下,也有一些方案专门提升针对性的防御,以提高基于编码的加密方案的安全性。而在公钥尺寸方面,进行了大量的研究工作。从采用一些好性质,结构紧凑的码,到公钥的细节设计或颠覆性设计,差错向量的控制等等。在本文的介绍中,McEliece版本的加密方案与Niederreiter版本的加密方案一直是研究的重点,两个版本的方案在一定程度上互相补充,Niederreiter在公钥尺寸上可以表现的更好,但是明文映射到差错向量,继而解密阶段反映射到明文,显然会使方案在效率上的表现不足。BBCRS公钥密码体制,可将其公钥的设计分别应用到McEliece版本与Niederreiter版本,这很好的避免了原始McEliece方案中置换矩阵带来的漏洞问题,很好的预防了传统的解码攻击,区分者攻击,密钥恢复攻击等等。但是在密码设计与密码分析的博弈中,方案安全性和敌手攻击手段总是水涨船高。

本文基于BBCRS公钥密码体制,提出了新的改进变体方案。经过以上的分析,我们针对性的解决了BBCRS公钥密码体制可能暴露的问题,并且在解密过程很好的避免了BBCRS的穷举搜索操作,提升了方案的执行效率。但是改进方案不是各个方面都有所提升,在公钥尺寸的问题上,我们的改进方案需要进行参数调试,整体平衡选择,才能使得安全性和公钥尺寸保持平衡,后续的工作依然是公钥尺寸问题的研究。

\section{展望}
在科学与社会的发展进程中,计算机,互联网的概念与应用,已经逐渐上升到云服务,物联网,数据中心,超算中心等实际的应用覆盖,将会更广而更深的影响人类社会的生活。网络资源,信息资源,逐渐扮演者个人或企业财产的角色,其重要性可想而知。与此同时,安全性的研究也在发展节奏当中,但是随着信息时代复杂性的上升,安全性的破坏者们反而手段更加丰富,安全性防御的压力就变得越来越大。安全性的基础研究,密码加密学与密码分析学,需要理论性的突破和快速的适用性以应对安全性的需求。

设计密码涉及基础学科很多,而且理论深度大,全民可以用密码,但是不能全民设计密码,这需要相关科学家与学者的共同努力。幸运的是,密码学的各个分支也都是持续发展着的,对称密码依然在适用的领域进行着快速的加解密运算,公钥密码系统也已经奠定了互联网协议安全的基础,而后量子时代,后量子密码也有四个主要的发展方向,其中本文涉及的基于编码的加密方案就是其中热门的研究领域之一。相信以后我们能很好的解决基于编码的加密方案现存的问题,以应对安全性的挑战。