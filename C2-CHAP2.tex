\chapter{预备知识与概念}
\label{chap1}
本章介绍基于编码的加密方案涉及到的有限域知识,编码知识等等。有限域理论是编码理论的一个重要数学基础,有限域上的元素性质,多项式性质更好的描述了码字在加解密中的作用,编码知识则是加密算法的基础,选择好性质的码是保证加密算法高效和安全的前提。

\section{有限域基础理论}
\begin{define}[有限域]
	在有限集合$\mathbf{F}$上定义了两个二元运算:加法 “$+$” 和乘法 “$\bullet$”,如果$(\mathbf{F}, +)$是交换群,$\mathbf{F}$的非零元素对乘法构成交换群,而且乘法对加法满足分配律,则称$(\mathbf{F}, +, \bullet)$是有限域,如果$\mathbf{F}$集合元素的个数为$q$,则记作$\mathbb{F}_q$或者$GF(q)$。
\end{define}

\begin{define}[本原元]
	一个数域$GF(q)$,具有最大阶的域元素为本原元,即本原元为$a$,则$a^d=1(mod~ q)$成立,其中$d=\psi(q)$,$\psi(q)$是欧拉函数。
\end{define}

\begin{define}[不可约多项式]
	有限域$\mathbb{F}$上定义的多项式集合$F[x]=\{f(x)|f(x)=a_nx^n + ... + a_1x + a_0,a_i \in \mathbb{F}, a_n \neq 0, n \geq 0\}$,$F[x]$中次数大于1的多项式$f(x)$不能写成两个低次多项式的乘积,称$f[x]$是$\mathbf{F}$上不可约多项式。
\end{define}

\begin{define}[极小多项式]
	设$\mathbb{F}_q$是一个含有$q$个元素的有限域,$\mathbb{F}_p$是$\mathbb{F}_q$的一个含有$p$个元素的子域,$\alpha \in \mathbb{F}_q$。$\mathbb{F}_p$上的以$\alpha$为根,首项系数为$1$,并且次数最低的多项式称为$\alpha$在$\mathbb{F}_p$上的极小多项式。这里$1$是$\mathbb{F}_p$的单位元。
\end{define}

\begin{define}[本原多项式]
	设$\mathbb{F}_(q^n)$是一个含有$q^n$个元素的有限域,$\mathbb{F}_q$是$\mathbb{F}_(q^n)$的一个含有$q$个元素的子域。设$\alpha \in \mathbb{F}_(q^n)$为$\mathbb{F}_(q^n)$的一个本原元。$\alpha$在$\mathbb{F}_q$上的极小多项式称为$\mathbb{F}_q$上的一个本原多项式。
\end{define}

\section{编码基础知识}

\subsection{汉明重量与汉明距离}

\subsection{线性码}

\subsection{生成矩阵与校验矩阵}

\subsection{解码算法}

\section{常见码的性质}

\subsection{GRS 码}

\subsection{Goppa 码}

\subsection{Gabidulin 码}

\section{本章小结}
