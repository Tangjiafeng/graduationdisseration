\chapter{预备知识与概念}
\label{chap1}
本章介绍基于编码的加密方案涉及到的有限域知识,编码知识等等。有限域理论是编码理论的一个重要数学基础,有限域上的元素,多项式更好的描述了码字在加解密中的计算过程,编码知识则是加密算法的理论依据,能够选择好性质的码是保证加密算法安全和高效的前提。

\section{有限域基础理论}
\begin{define}[有限域]
	在有限集合$\mathbf{F}$上定义了两个二元运算:加法 “$+$” 和乘法 “$\bullet$”,如果$(\mathbf{F}, +)$是交换群,$\mathbf{F}$的非零元素对乘法构成交换群,而且乘法对加法满足分配律,则称$(\mathbf{F}, +, \bullet)$是有限域,如果$\mathbf{F}$集合元素的个数为$q$,则记作$\mathbb{F}_q$或者$GF(q)$。简单起见,我们将二元域记作$\mathbb{F}$。
\end{define}

\begin{define}[本原元]
	一个数域$GF(q)$,具有最大阶的域元素为本原元,即本原元为$a$,则$a^d=1(mod~ q)$成立,其中$d=\psi(q)$,$\psi(q)$是欧拉函数。
\end{define}

\begin{define}[不可约多项式]
	有限域$\mathbb{F}$上定义的多项式集合$F[x]=\{f(x)|f(x)=a_nx^n + ... + a_1x + a_0,a_i \in \mathbb{F}, a_n \neq 0, n \geq 0\}$,$F[x]$中次数大于1的多项式$f(x)$不能写成两个低次多项式的乘积,称$f[x]$是$\mathbf{F}$上不可约多项式。
\end{define}

\begin{define}[极小多项式]
	设$\mathbb{F}_q$是一个含有$q$个元素的有限域,$\mathbb{F}_p$是$\mathbb{F}_q$的一个含有$p$个元素的子域,$\alpha \in \mathbb{F}_q$。$\mathbb{F}_p$上的以$\alpha$为根,首项系数为$1$,并且次数最低的多项式称为$\alpha$在$\mathbb{F}_p$上的极小多项式。这里$1$是$\mathbb{F}_p$的单位元。
\end{define}

\begin{define}[本原多项式]
	设$\mathbb{F}_(q^n)$是一个含有$q^n$个元素的有限域,$\mathbb{F}_q$是$\mathbb{F}_(q^n)$的一个含有$q$个元素的子域。设$\alpha \in \mathbb{F}_(q^n)$为$\mathbb{F}_(q^n)$的一个本原元。$\alpha$在$\mathbb{F}_q$上的极小多项式称为$\mathbb{F}_q$上的一个本原多项式。
\end{define}

\section{编码基础知识}
基于编码的加密方案的设计,需要考虑编码理论的相关定理,码的性质,纠错能力影响因素等,本小节做基本的概念介绍。在本文中介绍编码理论我们集中在二元域$\mathbb{F} = GF(2)$上。

\begin{define}[汉明重量与汉明距离]
	一个码字$\textbf{x}$的汉明距离定义为码字本身含有的非零位的个数,并表示为:$wt(\textbf{x})$。汉明距离指的是两个长度相同的码字$\textbf{x}, \textbf{y}$之间,位不同的总数,一般记作:$dist(\textbf{x}, \textbf{y})$。可以看出,$dist(\textbf{x}, \textbf{y}) = wt(\textbf{x} - \textbf{y})$。
\end{define}

\begin{define}[线性码]
	向量空间$\mathbb{F}^n$上的一个$k$维子空间定义为$[n, k]$线性码,记作:$\mathcal{C}$。另外,线性码$\mathcal{C}$的最小距离指的是,线性码中任意两个不同的码字之间的最小汉明距离,记作$d$。我们可以用$[n, k, d]$来表示最小距离为$d$的线性码$\mathcal{C}$,且最大纠错能力为$\lfloor(d - 1)\rfloor/2$。
\end{define}

\begin{define}[生成矩阵与校验矩阵]
	线性码$\mathcal{C}$的生成矩阵是指一个$k \times n$的矩阵$G$,$G$的行可以构成线性码$\mathcal{C}$的基。也就是说由矩阵$G$的行可以线性组合成线性码$\mathcal{C}$中所有的码字。矩阵$G$的系统形式,就是通过矩阵变换的前$k$列组成单位矩阵。线性码$\mathcal{C}$校验矩阵$H$是指生成矩阵的对偶形式,形状即$(n - k) \times n$。	
\end{define}

\begin{define}[解码算法]
	在基于编码的加密方案中,当我们选定一种$[n,k,d]$线性码$\mathcal{C}$时,都有一个相应的解码算法$D_{\mathcal{C}}$,完成纠正加入差错向量的码字的工作,也就是说,对任意的$\textbf{e} \in \mathbb{F}^n,~wt(\textbf{e}) < d/2,~\textbf{x} \in \mathcal{C}$,都有
	
	\centering $D_{\mathcal{C}}(\textbf{x} + \textbf{e}) = \textbf{x}$。
\end{define}

\section{常见码的构造}
编码理论指导我们寻找一些好码,使得信源信息经过编码后的,通过信道传输,在信道接收端可以实现自动纠错和检错。良好的纠错检错能力对基于编码的加密算法的作用是很关键的,应用表现好的码,就能减小方案中的码长,从而缩减公钥尺寸,极大的促进基于编码的加密方案的实行。本小节我们介绍几种常见的码,便于从中发现提升码性质的技术和方向。

\subsection{GRS 码}
GRS码,即广义Reed-Solomon码,因为其码结构紧凑的优势,在早期作为Goppa码的竞争者应用在基于编码的加密方案中。这是一种特殊的BCH码,BCH码是一种性质良好的循环码,首先循环码的定义是:

\begin{define}[循环码]
	设线性码$\mathcal{C}$,如果线性码$\mathcal{C}$的任意一个码字的循环移位还是一个码字,即当$a_0a_1···a_{n-1} \in \mathcal{C}$时,$a_{n-1}a_0a_1···a_{n-2} \in \mathcal{C}$,则称$\mathcal{C}$是一个循环码。
\end{define}

BCH码是由三位学者 (R. C. Bose, D. K. Ray-Chaudhuri, A. Hocquenghem)分别独立提出的,当码长不是很长时,纠错性能非常接近于理论值。BCH码构造方便,且编码和译码过程容易,非常具有研究价值。

\begin{define}[BCH码]
	设$\mathbb{F}_q$上的一个$r$维向量空间为$\mathbb{F}_{q^r}$,并且$1, \alpha, \alpha ^ 2, ... , \alpha ^ {r - 1}$是$\mathbb{F}_{q^r}$在$\mathbb{F}_q$上的一组基,则$\mathbb{F}_{q^r}$中的任意一个元素都可以唯一地表示为$1, \alpha, \alpha ^ 2, ... , \alpha ^ {r - 1}$的一个线性组合。令
	
	\centering $B_q(n, \delta, \alpha)=\{\mathbf{c}=c_0c_1c_2···c_{n-1} | \mathbf{c}H^\mathtt{T}\}$,其中$1 < \delta < n$,且
	\begin{equation}       %开始数学环境
	H =
	\left(                 %左括号
	\begin{array}{ccccc}   %该矩阵一共3列,每一列都居中放置
	1 & \alpha & \alpha^2 & ··· & \alpha^{n-1}\\  %第一行元素
    1 & \alpha^2 & (\alpha^2)^2 & ··· & (\alpha^2)^{n-1}\\  %第二行元素
    \vdots & \vdots & \vdots & \vdots & \vdots \\
    1 & \alpha^{\delta - 1} & (\alpha^{\delta - 1})^2 & ··· & (\alpha^{\delta - 1})^{n-1}\\
	\end{array}
	\right)                 %右括号
	\end{equation}
	\begin{flushleft}
        我们称$B_q(n, \delta, \alpha)$是码长为$n$并且设计距离为$\delta$的$q$元BCH码。我们对BCH码做一点推广,假设$b \geq 0$是一个非负整数,将矩阵$H$第一行中$\alpha$替换成$\alpha ^ b$,其它按照规律生成,则称$B_q(n, \delta, \alpha, b)$为广义BCH码。
	\end{flushleft}	
\end{define}

\begin{define}[RS码]
	设$q \geq 3$是一个素数的幂次方,码长为$q - 1$并且设计距离为$\delta$的$q$元BCH码$B_q(q - 1, \delta, \alpha)$称为$q$元Reed-Solomon码,简称$q$元RS码,记作$S(q - 1, \delta, \alpha)$,其中$\alpha$是$\mathbb{F}_q$的一个$q - 1$阶元素,即$\alpha$是$\mathbb{F}_q$的一个本原元。在广义BCH码的基础上,广义RS码也就记作$S(q - 1, \delta, \alpha, b)$。
\end{define}

广义的Reed-Solomon码与广义的BCH码的关系,如同RS码与BCH码的关系。

\subsection{Goppa 码}


\section{本章小结}
