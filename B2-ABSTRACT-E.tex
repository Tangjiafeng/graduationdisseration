\newpage
\vspace{-1cm}
\chapter*{\zihao{-2}\heiti{ABSTRACT}}
\vspace{-0.5cm}
The emergence of quantum computers has profoundly affected the development of various fields of computers. The important advantage of quantum computers is quantum parallel computing, which allows quantum computers to reach the level of computing power that classical computers cannot. As a result, many problems that classic computers cannot solve can be solved quickly by quantum computers. This is undoubtedly a huge improvement, but on the other hand, supercomputing power is also a threat to the security of the information age built by computers today. Specifically, in the field of cryptography, the security of the cryptographic algorithm is based on the computing power of the current computer. Under the allowed cracking cost, the computing power possessed by the computer cannot attack effectively the encryption scheme, and the cryptographic scheme is safe. Researches have shown that, under ideal conditions, existing quantum computers have placed classic cryptographic algorithms at serious risk. In order to resist the attacks brought by quantum computers in the future, a variety of anti-quantum encryption algorithms are under study. For example: Hash-based public key cryptosystem;  lattice-based public key cryptosystem; Multivariate Public Key Cryptosystems; code-based public key cryptosystem.

The above cryptosystem is also called post-quantum cryptography. So far, no quantum computer algorithm has been able to effectively attack it. This article focuses on code-based public key encryption schemes. The code-based public key encryption scheme was first proposed by McEliece in 1978\cite{Mceliece1978A}. The encryption process of the scheme is to treat the plaintext as a legal codeword and add a correctable error vector. When decrypting, the code structure is firstly decoded and corrected, and then resume the plaintext. Although the code-based public key encryption scheme is recognized to be resistant to quantum attacks, and has high security and implementation efficiency, the cryptographic technology cannot be widely applied on a large scale. This is mainly because the public key encryption has the problem that the public key is too large. The next focus of research is to reduce the public key size of the algorithm.

There are many variations of the McEliece scheme in encryption schemes based on coding theory\cite{Loidreau2001Weak}. In general, attempts at these variants always take advantage of two basic ways to enhance the security and performance of the cryptosystem. One is to reduce the size of the public key; the other is to improve the efficiency and error correction ability of the decoding algorithm. At the same time, high security levels are always important. In this paper, we propose a new public key construction method according to the idea of the BBCRS scheme\cite{Baldi2011EnhancedPK}. This improvement enhances the error correction capability of the code and can better protect the secret key structure. We also discussed some known attacks in detail in the BBCRS program, and the results show that our new solution is safe under the currently known attack methods.

\hspace{-0.5cm}
{\sihao{\textbf{Keywords:}}} \textit{Post-quantum cryptography;\, code-based cryptography;\, McEliece cryptosystem;\, public key encryption.
}


































