\chapter{基于BBCRS公钥密码体制的改进}
基于编码的加密方案,后续的改进工作主要集中在缩小公钥尺寸,提升解码效率两大方面。BBCRS公钥密码体制在安全性提升的同时,其Niederreiter版本保证了公钥尺寸与原始Niederreiter等同;在安全等级相同的情况下,加解密效率高于基于传统数论的RSA。为了缩小公钥尺寸,相关学者对各种能应用到McEliece加密方案的码进行了大量研究,比如低密度奇偶校验码,但是经过密码分析者的分析工作,这些码应用到McEliece加密方案中是不安全的,特别是改进的信息集解码攻击极易对这么设计产生威胁。但是BBCRS公钥密码体制的公钥设计,在一定程度上可以对码作出妥协,比如可以选择结构更紧凑的GRS码,在安全性上依然与应用Goppa的McEliece加密方案基本等同。

BBCRS公钥密码体制,在公钥的设计上是敢于突破的,不管是McEliece及其变体方案,还是Niederreiter及其变体方案,在公钥的设计上总是类似的,公钥与私钥之间总是保留着置换的相等关系。BBCRS选择直接将置换矩阵替换为其它形式的矩阵,在基于编码的公钥构造上提供了一些思路。本章内容,就是根据这种思路,提供了另一种公钥构造方式,该种方式可以达到BBCRS公钥密码体制中描述的公钥与私钥之间不再是置换相等的关系,从而在安全性上表现突出,在加解密表现上避免了BBCRS公钥密码体制出现的情况,是一种可取的改进方式。

\section{研究动机}
Marco Baldi等人在BBCRS公钥密码体制发表后,总结到方案采用GRS码既可以保证安全性的提升,又能带来公钥尺寸的减小。在同一种安全性级别,在加解密的操作复杂度显著低于RSA公钥密码体制。但是在2015年,有密码分析学者提出的密钥恢复攻击可以在多项式时间内攻破采用GRS码的BBCRS公钥密码体制。具体来说,密码分析学者提出的攻击方案是针对矩阵$T$平均行重或列重范围在$[1,2]$的情况,利用矩阵$T$行列重量分布的特点,将BBCRS公钥密码体制的安全性降低到利用区分者攻击手法可恢复私钥的安全性。

如此以来,在BBCRS公钥密码体制的基础上,规避暴露出来的弱点,就是待解决的主要问题。首先就是关于码的选择,BBCRS公钥密码体制推荐使用GRS码,以便更好的利用GRS码结构紧凑的特性。但是这也成为后续攻击手段的切入点,如果继而采用原始的Goppa码,就必须提升方案的信息利用率,不然在纠错能力$(\lfloor \frac{t}{m} \rfloor)$已经有所减少的情况下,对通信效率将是较大的影响。其次,方案在解密过程中,不仅要考虑中间结果超过码的最大纠错能力,还要考虑消除$\mathbf{e}R \neq \mathbf{0}$带来的穷举搜索的重复操作。另外,在加密过程中,随机选取的差错向量,重量已经有所减少,而且要满足$(\mathbf{a}_1 + \mathbf{a}_2 + ··· + \mathbf{a}_w) \cdot \mathbf{e}^\mathtt{T} = \mathbf{0}.$,这也带来了威胁较大的子码漏洞问题,虽然BBCRS公钥密码体制提出了解决之道,但是如果能放开这样的限制条件,或者改为其它的限制,方案的设计会更合理。

BBCRS公钥密码体制替换原始McEliece加密方案的置换矩阵,不仅消除了公钥与私钥之间置换相等的关系,而且矩阵$Q$的组成让我们有更多的细粒度的参数设定,比如$w$的值决定集合中矩阵的个数,$z$的值决定矩阵$R$的秩,BBCRS公钥密码体制Niederreiter版本要存储的公钥尺寸,和矩阵$R$的秩是息息相关的。这样的设计可以让我们根据不同的码的特点,空间效率等因素选择最优的参数来实现加解密。本文在设计基于BBCRS的改进方案时,也是将置换矩阵替换成多个矩阵计算结果的形式。BBCRS公钥密码体制在差错向量上的类似于陷门设定,在后续解密简化了步骤,这些都为新方案提供了思想指导。

\section{改进方向}
考虑到设计的可行性,本文主要对McEliece版本的BBCRS公钥密码体制进行公钥构造形式上的改进,参考以往McEliece方案的变体,随机非奇异矩阵$S$的作用基本不变,混淆生成矩阵为随机的矩阵,所以在新方案中,我们也保留这一左乘的形式。公钥的构造关键,就是原始McEliece方案的置换矩阵$P$这一部分,明文编码成含有错误码字,在进一步混淆私钥结构的过程中,就要考虑码字错误位数的扩散问题,置换矩阵因为每行每列都只有一个非零位,所以在计算$\mathbf{e}P^{-1}$的时候不会出现错误位数大于$\mathbf{e}$的汉明重量,即仍然可以正确的进行译码。回顾BBCRS公钥密码体制在这部分的考虑,其一是对差错向量有一定的约束,其二是对矩阵$T$在平均行重和列重做了规定,而且分析了解码失败的概率。于是在新公钥构造方式中,如何使约束成本最小化,混淆作用最大化就成了关键。

在BBCRS公钥密码体制的改进方向上,基本上确定矩阵$Q$是安全性和效率的关键部分,如何设计一个陷门,能把在加密过程中加入的混淆作用,在解密的时候消除掉是重中之重。首先我们想到可逆矩阵,可逆矩阵无疑是隐藏私钥矩阵的利器,但是我们不能忽略一个可逆矩阵与差错向量的计算结果,很明显,差错向量对汉明重量的要求,即码的纠错能力将毫无意义,因为可逆矩阵的扩散作用非常难以控制。

\section{方案设计}


\section{安全性分析}


\section{效率分析}


\section{本章小结}
