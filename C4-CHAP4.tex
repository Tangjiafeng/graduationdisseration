\chapter{基于BBCRS公钥密码体制的改进}
基于编码的加密方案,后续的改进工作主要集中在缩小公钥尺寸,提升解码效率两大方面。BBCRS公钥密码体制在安全性提升的同时,其Niederreiter版本保证了公钥尺寸与原始Niederreiter等同;在安全等级相同的情况下,加解密效率高于基于传统数论的RSA。为了缩小公钥尺寸,相关学者对各种能应用到McEliece加密方案的码进行了大量研究,比如低密度奇偶校验码,但是经过密码分析者的分析工作,这些码应用到McEliece加密方案中是不安全的,特别是改进的信息集解码攻击极易对这么设计产生威胁。但是BBCRS公钥密码体制的公钥设计,在一定程度上可以对码作出妥协,比如可以选择结构更紧凑的GRS码,在安全性上依然与应用Goppa的McEliece加密方案基本等同。

BBCRS公钥密码体制,在公钥的设计上是敢于突破的,不管是McEliece及其变体方案,还是Niederreiter及其变体方案,在公钥的设计上总是类似的,公钥与私钥之间总是保留着置换的相等关系。BBCRS选择直接将置换矩阵替换为其它形式的矩阵,在基于编码的公钥构造上提供了一些思路。本章内容,就是根据这种思路,提供了另一种公钥构造方式,该种方式可以达到BBCRS公钥密码体制中描述的公钥与私钥之间不再是置换相等的关系,从而在安全性上表现突出,在加解密表现上避免了BBCRS公钥密码体制出现的情况,是一种可取的改进方式。

\section{研究动机}
Marco Baldi等人在BBCRS公钥密码体制发表后,总结到方案采用GRS码既可以保证安全性的提升,又能带来公钥尺寸的减小。在同一种安全性级别,在加解密的操作复杂度显著低于RSA公钥密码体制。但是在2015年,有密码分析学者提出可以在多项式时间内攻破采用GRS码的BBCRS公钥密码体制。

\section{改进方向}


\section{方案设计}


\section{安全性分析}


\section{效率分析}


\section{本章小结}
