\vspace{-2.5cm}
\chapter*{\zihao{2}\heiti{摘~~~~要}}
\vspace{-1cm}
量子计算机的出现,深刻影响着计算机各个领域的发展。量子计算机的重要优越性就是量子并行计算,这使得量子计算机可以达到经典计算机无法达到的算力水平。于是,很多经典计算机无法解决的问题,量子计算机就能很快解决。这无疑是巨大的进步,但是从另一方面来看,超级计算能力对如今计算机构建的信息时代安全本身也是一种威胁。具体来看,在密码学领域,密码算法的安全性是基于当前计算机的计算能力的,在允许的破解成本下,计算机所拥有的算力无法破解密码,就说密码算法是安全的。研究表明,在理想情况下,现有的量子计算机,已经使经典的密码算法处于严重威胁之中。为了抵抗未来量子计算机带来的攻击,有多种抗量子加密算法已经处于研究之中。比如:基于Hash函数的公钥密码体制;基于格问题的公钥密码体制;基于多变量问题的公钥密码体制;基于编码问题的公钥密码体制。

如上的密码体制又叫做后量子密码,迄今为止,还没有量子计算机算法对其能进行有效的攻击。本文主要讨论的是基于编码的公钥加密方案。基于编码的公钥加密方案的是在1978年由McEliece首次提出的\cite{Mceliece1978A},该方案加密过程是将明文当作合法码字并加入可以纠正的错误向量,解密时根据码结构先进行有效译码纠错,再恢复明文。虽然目前基于编码的公钥加密方案公认是的可以抵抗量子攻击,且具有较高的安全性和实现效率,但是该密码技术仍无法大规模广泛应用。这主要是因为基于编码的公钥加密方法都存在这公钥过大的问题。后续重点的研究方向就是缩小算法的公钥尺寸。

在基于编码理论的加密方案中,有很多McEliece方案的变体\cite{Loidreau2001Weak}。一般来说,这些变体的尝试总是利用两种基本方式来增强密码系统的安全和性能。其中一个是减小公钥的大小;另一个是提高解码算法的效率和纠错能力。与此同时,安全级别是一直追求的目标。在本文中,我们按照BBCRS方案\cite{Baldi2011EnhancedPK}的思想提出了一种新的公钥构造方式。这种改进增强了编码的纠错能力,并且可以更好地保护密钥结构。 我们还在BBCRS方案中详细讨论了一些已知的攻击,结果表明我们的新密钥生成方案在当前已知的攻击手段下是安全的。


\hspace{-0.5cm}
\sihao{\heiti{关键词:}} \xiaosi{抗量子密码学,基于编码的加密学,McEliece加密方案,公钥加密}
